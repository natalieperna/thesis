Agda is a dependently-typed programming language and theorem prover,
supporting proof construction in a functional programming style.
Due to its incredibly flexible concrete syntax and support for Unicode
identifiers, Agda can be used to construct elegant and expressive
proofs in a format that is understandable even to those unfamiliar
with the tool.

However, the semantics of Agda is lacking resource guarantees
of the kind that Haskell programmers are used to with lazy evaluation,
where multiple uses of function arguments and let-bound variables
still result in the corresponding expressions to be evaluated at most
once.
With the current compiler backends of Agda,
a mathematically-natural way to structure programs
therefore frequently results in inefficient compiled programs,
where the run-time complexity can be exponentional in cases where
corresponding Haskell code executes in linear time.

This makes a highly-optimised compiler backend a particularly
essential tool for practical development with Agda.
The main contributions of this thesis are a series of compiler
optimisations that inlines simple projections, removes some
expressions with trivial evaluations that can be statically inferred,
and reduces the need for repeated evaluations of the same expressions
by increasing sharing.

We developed transformations that focus on the inherent ``loss'' of sharing
that is frequently the result of compiling Agda programs. Where an Agda
developer may imagine that value sharing should exist in the generated Haskell
code, it often does not.
We present several optimising transformations that re-introduce some of this
``lost'' sharing without affecting the type-theoretic semantics, then apply
these optimisations to several typical Agda applications to examine the
memory allocation and execution time effects.

In measuring the effects of these optimisations
on Agda code we show that overall improvements in runtime on the order
of 10-20\% are possible.
We hope that the development and discussion of these optimisations is
useful to the Agda developer community, and may be helpful for future
contributors interested in implementing new optimisations for Agda.
