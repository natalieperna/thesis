%________________________________________________________________________
%_________________     BEGIN LINEAR NOTATION MACROS    __________________
%________________________________________________________________________

% Linear Notation: Standard
\newcommand{\lnotation}[4]{
	\def\third:{#3} 
	\def\possiblyone:{} 
	\def\possiblytwo:{~}
	\def\possiblythree:{ }
	\def\divide{\;#1\hspace*{-0pt}( #2\; \mid: \; #4 \, )}
	\def\nodivide{\;#1\hspace*{-0pt}( #2\;\mid\; #3\;:\;#4 \, )}
	\ifx\third\possiblyone\divide
		\else\ifx\third\possiblytwo\divide
		\else \ifx\third\possiblythree\divide
		\else \nodivide\fi\fi\fi}

% Linear Notation: Standard (Big Parentheses)
\newcommand{\biglnotation}[4]{
	\def\third:{#3} 
	\def\possiblyone:{} 
	\def\possiblytwo:{~}
	\def\possiblythree:{ }
	\def\divide{\;#1\hspace*{-0pt}\big( #2\; \mid: \; #4 \, \big)}
	\def\nodivide{\;#1\hspace*{-0pt}\big( #2\;\mid\; #3\;:\;#4 \, \big)}
	\ifx\third\possiblyone\divide
		\else\ifx\third\possiblytwo\divide
		\else \ifx\third\possiblythree\divide
		\else \nodivide\fi\fi\fi}

% Linear Notation: Standard (Extra Big Parentheses)
\newcommand{\bigglnotation}[4]{
	\def\third:{#3} 
	\def\possiblyone:{} 
	\def\possiblytwo:{~}
	\def\possiblythree:{ }
	\def\divide{\;#1\hspace*{-0pt}\bigg( #2\; \mid: \; #4 \, \bigg)}
	\def\nodivide{\;#1\hspace*{-0pt}\bigg( #2\;\mid\; #3\;:\;#4 \, \bigg)}
	\ifx\third\possiblyone\divide
		\else\ifx\third\possiblytwo\divide
		\else \ifx\third\possiblythree\divide
		\else \nodivide\fi\fi\fi}
 
% Linear Notation: Standard (No Ending Parentheses)
\newcommand{\wplnotation}[4]{
	\def\third:{#3} 
	\def\possiblyone:{} 
	\def\possiblytwo:{~}
	\def\possiblythree:{ }
	\def\divide{\;#1\hspace*{-0pt}\bigg( #2\; \mid: \; #4 \, }
	\def\nodivide{\;#1\hspace*{-0pt}\bigg( #2\;\mid\; #3\;:\;#4 \, }
	\ifx\third\possiblyone\divide
		\else\ifx\third\possiblytwo\divide
		\else \ifx\third\possiblythree\divide
		\else \nodivide\fi\fi\fi}

% Linear Notation: David Gries
\newcommand{\grieslnotation}[4]{
	\def\third:{#3} 
	\def\possiblyone:{} 
	\def\possiblytwo:{~}
	\def\possiblythree:{ }
	\def\divide{(#1 #2\; \mid : \; #4 \, )}
	\def\nodivide{(#1 #2\;\mid\; #3\;:\;#4 \, )}
	\ifx\third\possiblyone\divide
		\else\ifx\third\possiblytwo\divide
		\else \ifx\third\possiblythree\divide
		\else \nodivide\fi\fi\fi}

%________________________________________________________________________
%_________________      END LINEAR NOTATION MACROS     __________________
%________________________________________________________________________

%________________________________________________________________________
%_________________       BEGIN SET NOTATION MACROS     __________________
%________________________________________________________________________

% Set Notation: David Gries
\newcommand{\griesset}[3]{
	\def\second:{#2} 
	\def\possiblyone:{} 
	\def\possiblytwo:{~}
	\def\possiblythree:{ }
	\def\divide{\{#1\; \mid : \; #3 \, \}}
	\def\nodivide{\{#1\;\mid\; #2\;:\;#3 \, \}}
	\ifx\second\possiblyone\domvide
		\else \ifx\second\possiblytwo\divide
		\else \ifx\second\possiblythree\divide
		\else \nodivide\fi\fi\fi}

% Set Notation: Standard Enumeration
\newcommand{\set}[1]{\{#1\}}

% Set Notation: Standard Comprehension
\newcommand{\sets}[2]{\{#1\; \mid \; #2\}}

% Set Notation: Standard (Big Parentheses)
\newcommand{\bigset}[2]{\big\{#1\; \mid \; #2\big\}}

% Set Notation: Standard (Extra Big Parentheses)
\newcommand{\biggset}[2]{\bigg\{#1\; \mid \; #2\bigg\}}

%________________________________________________________________________
%_________________        END SET NOTATION MACROS      __________________
%________________________________________________________________________

%________________________________________________________________________
%_________________        BEGIN STRUCTURE MACROS       __________________
%________________________________________________________________________

% Common Calligraphic Names
\newcommand{\C}{{\cal{C}}}
\newcommand{\F}{{\cal{F}}}
\newcommand{\II}{{\cal{I}}}
\newcommand{\M}{{\cal{M}}}
\newcommand{\N}{{\cal{N}}}
\newcommand{\R}{{\cal{R}}}
\newcommand{\Rel}{{\cal{R}}el}
\newcommand{\T}{{\cal{T}}}
\newcommand{\Z}{{\cal{Z}}}
\newcommand{\one}[1]{1_{#1}}

% Semi-Group
\newcommand{\semigroup}[2]{\big(#1, #2 \big)}

% Monoid
\newcommand{\monoid}[3]{\big(#1, #2, #3 \big)}

% Semi-Rings
\newcommand{\nsemiring}[3]{\big(#1, #2, #3\big)}
\newcommand{\semirng}[4]{\big(#1, #2, #3, #4\big)}
\newcommand{\semiring}[5]{\big(#1, #2, #3, #4, #5\big)}

% Group
\newcommand{\group}[4]{\big(#1, #2, #3, #4\big)}

% Semi-Module
\newcommand{\semimodule}[3]{\big(_{#1}#2, #3\big)}

% Relational Lattice
\newcommand{\latticeRel}[2]{\big(#1, #2 \big)}

% Algebraic Lattice
\newcommand{\latticeAlg}[3]{\big(#1, #2,  #3\big)}

%________________________________________________________________________
%_________________         END STRUCTURE MACROS        __________________
%________________________________________________________________________

%________________________________________________________________________
%_________________    BEGIN PROOF ENVIRONMENT MACROS   __________________
%________________________________________________________________________

% Spacing
\newlength{\interligne}

% Begin and End Basic Proof Enviroment
\newcommand{\Beginproof}{\dimen123=\linewidth \dimen124=\linewidth
	\advance\dimen123 by -25mm \advance\dimen124 by -5mm
	\advance\dimen123 by -\parindent \advance\dimen124 by -\parindent
	\setlength{\interligne}{\baselineskip}
	\setlength{\baselineskip}{1.2\baselineskip}
    	\begin{tabbing}
    		\hspace*{\parindent}\= \hspace*{5mm}\= \kill \+ \kill}
			\newcommand {\Endproof}
		{\end{tabbing}
    \setlength{\baselineskip}{\interligne}}

% Comment
\newcommand{\com}[1]{\> \hspace*{15mm}
	$\langle$~\parbox[t]{\dimen123}{ #1 $\rangle$}\\}

% Coloured Comment
\newcommand{\comc}[1]{\> \hspace*{15mm}
	$\langle$~\parbox[t]{\dimen123}{ \textcolor{red}{#1} $\rangle$}\\}

% Predicate
\newcommand{\pred}[1]{\>\parbox[t]{\dimen124}{#1}\\}

% Comment Separator
\newcommand{\hsep}{\quad\&\quad}


% Begin and End Itemized/Enumerated Proof Enviroment
% Use \Beginproofitem when a proof begins the text of an \item 
% and when we want the proof to be indented.
\newcommand{\Beginproofitem}{\dimen123=\linewidth \dimen124=\linewidth
	\advance\dimen123 by -20mm \advance\dimen124 by -5mm
	\advance\dimen124 by -\parindent
    	\begin{tabbing}
    		\hspace*{5mm}\= \kill}
			\newcommand {\Endproofitem}
	{\end{tabbing}}

% Tabbing
% Gives the same indentation as \Beginproof and \dspec, 
% but there is no predefined tabs
\newcommand {\Begingtabin}{
	\begin{tabbing}
    	\hspace*{\parindent}\= \kill \+ \kill}
		\newcommand {\Endtabin}
	{\end{tabbing}}

% Begin and End Basic Program Specification Enviroment
\newcommand{\Begingspec}{
	\begin{tabbing}
    	\hspace*{\parindent}\= \hspace*{5mm}\=\hspace*{5mm}\=\hspace*{5mm}\=
			\hspace*{5mm}\=\hspace*{5mm} \kill \+ \kill}
		\newcommand {\Endspec}
	{\end{tabbing}}

% Begin and End Itemized/Enumerated Program Specification Enviroment
% Use \Beginproofitem when a proof begins the text of an \item 
% and when we do not want that the specification be indented
\newcommand {\Beginspecitem}{
	\begin{tabbing}
    	\hspace*{5mm}\=\hspace*{5mm}\=\hspace*{5mm}\=
    	\hspace*{5mm}\=\hspace*{5mm} \kill}
		\newcommand {\Endspecitem}
	{\end{tabbing}}

% Multiline Proof Formulas
\newcommand{\nln}{@{}l@{}}
\newenvironment{displaymany}{\[ \begin{array}{\nln}}{\end{array} \]}

%________________________________________________________________________
%_________________     END PROOF ENVIRONMENT MACROS    __________________
%________________________________________________________________________

%________________________________________________________________________
%_________________         BEGIN EQUATION MACROS       __________________
%________________________________________________________________________

% When ShowEqSourceStructure is at 1, it shows the argument of \eqfrom
\newcommand{\ShowEqSourceStructure}{1}

\newcommand{\eqname}[1]{\textcolor{black}{\mbox{#1}}}

\newcommand{\eqfrom}[1]{
	\ifthenelse{\ShowEqSourceStructure=1}
	{ \quad {\textcolor{red}{\mbox{#1}}}} {\quad} }

%________________________________________________________________________
%_________________          END EQUATION MACROS        __________________
%________________________________________________________________________

%________________________________________________________________________
%_________________          BEGIN LOGIC MACROS         __________________
%________________________________________________________________________

% Logical Connectives: Constants
\newcommand{\true}{\textsf{true}}
\newcommand{\false}{\textsf{false}}

% Logical Connectives: Unary
\newcommand{\Not}{\neg}

% Logical Connectives: Binary	 
\newcommand{\Or}{\mathrel{\vee}}
\newcommand{\Ors}{\;\Or\;}
\newcommand{\AAnd}{\mathrel{\wedge}}
\newcommand{\nAnd}{\;\AAnd\;}
\newcommand{\Imp}{\mathrel{\rightarrow}}
\newcommand{\Impl}{\mathrel{\leftarrow}}
\newcommand{\Iff}{\mathrel{\leftrightarrow}}
\newcommand{\mOr}{\textrm{\ or\ }}  
\newcommand{\mAnd}{\textrm{\ and\ }} 
\newcommand{\mImp}{\;\Longrightarrow\;}  
\newcommand{\mImpl}{\;\Longleftarrow\;}  
\newcommand{\mIff}{\;\Longleftrightarrow\;} 
\newcommand{\dIff}{\mathrel{\stackrel{\scriptscriptstyle
	\mbox{\tiny{\texsf{def}}}}{\Iff}}}

%________________________________________________________________________
%_________________           END LOGIC MACROS          __________________
%________________________________________________________________________

%________________________________________________________________________
%_________________           BEGIN SET MACROS          __________________
%________________________________________________________________________

% Common Sets
\newcommand{\ONE}{1\!\!1}
\newcommand{\Prime}{\mathbb{P}}
\newcommand{\Bool}{\mathbb{B}}
\newcommand{\Nat}{\mathbb{N}}
\newcommand{\Real}{\mathbb{R}}
\newcommand{\Integer}{\mathbb{Z}}
\newcommand{\STtop}{\textsf{U}}
\newcommand{\STbot}{\emptyset}
\newcommand{\STpowerset}[1]{{\cal{P}}(#1)}

% Operators: Unary
\newcommand{\STcomplement}[1]{\overline{#1}}
\newcommand{\STcplOP}{\overline{\phantom{Y}\,}}

% Operators: Binary
\newcommand{\STleq}{\subseteq}
\newcommand{\STlt}{\subset}
\newcommand{\STgeq}{\supseteq}
\newcommand{\STgt}{\supset}
\newcommand{\STjoin}{\; \cup \;}
\newcommand{\STmeet}{\; \cap \;}

% Closures
\newcommand{\rtclosure}[1]{#1^*{}}	
\newcommand{\rtc}{\rtclosure}
\newcommand{\tclosure}[1]{#1^+{}}	
\newcommand{\tcl}{\tclosure}

%________________________________________________________________________
%_________________            END SET MACROS           __________________
%________________________________________________________________________

%________________________________________________________________________
%_________________    BEGIN REALTION ALGEBRA MACROS    __________________
%________________________________________________________________________

% Common Relations
\newcommand{\RAtop}{\mathbb{L}}
\newcommand{\RAid}{\mathbb{I}}
\newcommand{\RAdi}{\RAcomplement{\RAid}}
\newcommand{\RAbot}{\emptyset}

% Operators: Unary
\newcommand{\RAcomplement}[1]{\overline{#1}}
\newcommand{\RAcplOP}{\overline{\phantom{Y}\,}}
\newcommand{\internalconverse}[1]{#1^{\mkern-1mu{}{\raise0.1ex\hbox{\tiny$\smallsmile\,$}}}\kern-0.1em{}}
\newcommand{\RAconverse}[1]{\internalconverse{#1}}
\newcommand{\RAconverseOP}{\RAconverse{}\,}

% Operators: Binary
\newcommand{\RAcomp}{\mathop{\kern-.5pt\raise.3ex\hbox{\footnotesize\rm;}}}
\newcommand{\RAleq}{\sqsubseteq}
\newcommand{\RAgeq}{\sqsupseteq}
\newcommand{\RAjoin}{\; \sqcup \;}
\newcommand{\RAmeet}{\; \sqcap \;}
\newcommand{\RAcompose}[2]{#1\,\mbox{$\RAcomp$}\, #2}
\newcommand{\RAcomposet}[3]{\RAcompose{\RAcompose{#1}{#2}} {#3}}
\newcommand{\RAcomposeq}[4]{\RAcompose{\RAcompose{#1}{#2}} {\RAcompose {#3}{#4}}}
\newcommand{\RAcomposec}[5]{\RAcompose{\RAcomposeq{#1}{#2}{#3}{#4}} {#5}}
\newcommand{\RAcomposes}[6]{\RAcompose{\RAcomposet{#1}{#2}{#3}} {\RAcomposet {#4}{#5} {#6}}}
\newcommand{\rressymbol}{\backslash}
\newcommand{\RArightresidual}[2]{{#1}\rressymbol{#2}}
\newcommand{\lressymbol}{/}
\newcommand{\RAleftresidual}[2]{{#1}\lressymbol{#2}}
\newcommand{\RAsyq}[2]{\textsf{syq}(#1,#2)}

% Others
\newcommand{\RAsubset}{\sqsubset}
\newcommand{\elmtspecial}{\sharp}

% Types
\newcommand{\typerel}[2]{#1 \leftrightarrow #2}
\newenvironment{arra}[2][]{\begin{array}[#1]{@{}#2@{}}}{\end{array}} 
\newenvironment{tabul}[2][]{\begin{tabular}[#1]{@{}#2@{}}}{\end{tabular}} 

%________________________________________________________________________
%_________________     END REALTION ALGEBRA MACROS     __________________
%________________________________________________________________________

%________________________________________________________________________
%_________________       BEGIN HOARE LOGIC MACROS      __________________
%________________________________________________________________________

% Guards
\newcommand{\guardsymb}{\lceil\!\:\!\!\rfloor}
\newcommand{\guardif}[2]{\ifs \quad #1 \longrightarrow #2}
\newcommand{\guardb}[2]{\guardsymb\; \quad #1 \longrightarrow #2}
\newcommand{\guarde}{\fis}

% Keywords
\newcommand{\gives}{\longrightarrow}
\newcommand{\derive}[2]{\overset{#2}{\underset{#1}{\longrightarrow}}}
\newcommand{\gderive}[1]{\overset{#1}{\underset{G}{\longrightarrow}}}
\newcommand{\nocc}[2]{\neg\occurs('#1',\; '#2')}
\newcommand{\odd}[1]{\oddd(#1)}
\newcommand{\even}[1]{\evenn(#1)}
\newcommand{\hoare}[3]{\{#1\}\;#2\;\{#3\}}
\newcommand{\lpthoare}[4]{
	\begin{tabbing}
 		$\{#1\}\;\mbox{\bp{$#2$}}\;$\=$\{#3$\\
        \> $\;\;#4\;\}$
	\end{tabbing}
}
\newcommand{\choare}[3]{\mbox{\bp{$\{#1\}$}}\;#2\;\mbox{\mpoint{$\{#3\}$}}}
\newcommand{\massign}[2]{\mbox{$#1\;:=\;#2$} }
\newcommand{\subst}[2]{\left[\mbox{$#1\;:=\;#2$}\right] }

%________________________________________________________________________
%_________________        END HOARE LOGIC MACROS       __________________
%________________________________________________________________________

%________________________________________________________________________
%_________________      BEGIN COMMON SYMBOL MACROS     __________________
%________________________________________________________________________

% Functions / Relations
\newcommand{\dom}[1]{\textsf{dom\ }\!(#1)}	
\newcommand{\ran}[1]{\textsf{ran\ }\!(#1)}	
\newcommand{\tuple}[1]{(#1)}		

% Sets
\newcommand{\abs}[1]{|#1|}
\newcommand{\length}[1]{|#1|}
\newcommand{\card}[1]{\# \! #1}
\newcommand{\sizeset}[1]{\mid  \! #1 \! \mid}
\newcommand{\cardp}[1]{\# \!\left(#1\right)}

% Vectors
\newcommand{\zero}{\hat{0}}
\newcommand{\un}{\hat{1}}
\newcommand{\lator}{\;\mbox{\scalebox{0.5}{$\nabla$}}\;}
\newcommand{\latand}{\; \mbox{\scalebox{0.5}{$\triangle$}}\;}

% Prime
\newcommand{\p}[1]{\mbox{$#1'$}}

% Definitions
\newcommand{\deq}{\spaces{\stackrel{\textrm{def}}{=}}}
\newcommand{\defiff}{\spaces{\stackrel{\textrm{def}}{\mIff}}}

% Equivalences
\newcommand{\sequiv}[2]{#1\approx #2}
\newcommand{\equivwrt}[3]{#1\approx_{#2} #3}
\newcommand{\class}[1]{\left[#1\right]}

%________________________________________________________________________
%_________________       END COMMON SYMBOL MACROS      __________________
%________________________________________________________________________

%________________________________________________________________________
%_________________         BEGIN KEYWORD MACROS        __________________
%________________________________________________________________________

% Fonts and Colours
\newcommand{\keyw}[1]{{\textsf{#1}}}
\newcommand{\ckeyw}[1]{\mbox{\textcolor{blue}{\textsf{#1}}}}

% Spacing
\newcommand{\rspace}[1]{#1\ }
\newcommand{\lspace}[1]{\ #1}
\newcommand{\spaces}[1]{\,#1\,}

% Keywords
\newcommand{\oddd}{\keyw{odd}}
\newcommand{\evenn}{\keyw{even}}
\newcommand{\occurs}{\keyw{occurs}}
\newcommand{\aborts}{\keyw{abort}}
\newcommand{\begins}{\rspace{\keyw{begin}}}
%\newcommand{\ends}{\lspace{\keyw{end}}}
\newcommand{\divs}{\spaces{\keyw{div}}}
\newcommand{\dos}{\spaces{\keyw{do}}}
\newcommand{\elses}{\spaces{\keyw{else}}}
\newcommand{\skips}{\spaces{\keyw{skip}}}
\newcommand{\error}{\keyw{error}}
\newcommand{\fis}{\lspace{\keyw{fi}}}
\newcommand{\fors}{\spaces{\keyw{for}}}
\newcommand{\ifs}{\rspace{\keyw{if}}}
\newcommand{\mods}{\spaces{\keyw{mod}}}
\newcommand{\notts}{\rspace{\keyw{not}}}
\newcommand{\ods}{\lspace{\keyw{od}}}
\newcommand{\procs}{\rspace{\keyw{proc}}}
\newcommand{\thens}{\spaces{\keyw{then}}}
\newcommand{\vars}{\rspace{\keyw{var}}}
\newcommand{\whiles}{\rspace{\keyw{while}}}
\newcommand{\wrts}{\spaces{\keyw{wrt}}}
\newcommand{\revs}{\spaces{\keyw{rev}}}

% Coloured Keywords
\newcommand{\coddd}{\ckeyw{odd}}
\newcommand{\cevenn}{\ckeyw{even}}
\newcommand{\coccurs}{\ckeyw{occurs}}
\newcommand{\caborts}{\ckeyw{abort}}
\newcommand{\cbegins}{\rspace{\ckeyw{begin}}}
\newcommand{\cends}{\lspace{\ckeyw{end}}}
\newcommand{\cdivs}{\spaces{\ckeyw{div}}}
\newcommand{\cdos}{\spaces{\ckeyw{do}}}
\newcommand{\celses}{\spaces{\ckeyw{else}}}
\newcommand{\cskips}{\spaces{\ckeyw{skip}}}
\newcommand{\cerror}{\ckeyw{error}}
\newcommand{\cfis}{\lspace{\ckeyw{fi}}}
\newcommand{\cfors}{\spaces{\ckeyw{for}}}
\newcommand{\cifs}{\rspace{\ckeyw{if}}}
\newcommand{\cmods}{\spaces{\ckeyw{mod}}}
\newcommand{\cnotts}{\rspace{\ckeyw{not}}}
\newcommand{\cods}{\lspace{\ckeyw{od}}}
\newcommand{\cprocs}{\rspace{\ckeyw{proc}}}
\newcommand{\cthens}{\spaces{\ckeyw{then}}}
\newcommand{\cvars}{\rspace{\ckeyw{var}}}
\newcommand{\cwhiles}{\rspace{\ckeyw{while}}}
\newcommand{\cwrts}{\spaces{\ckeyw{wrt}}}
\newcommand{\crevs}{\spaces{\ckeyw{rev}}}

%________________________________________________________________________
%_________________          END KEYWORD MACROS         __________________
%________________________________________________________________________

%________________________________________________________________________
%_________________     BEGIN COVERT CHANNEL MACROS     __________________
%________________________________________________________________________

% Sets
\newcommand\Data{\mathbb{D}}

% Agents
\newcommand{\AgentOne}{\textrm{agent $A$}\@\xspace}
\newcommand{\AgentTwo}{\textrm{agent $B$}\@\xspace}
\newcommand{\Monitor}{\textrm{$M$}\@\xspace}

% Information
\newcommand{\CInfo}{\Phi_C}
\newcommand{\OInfo}{\Phi_O}

% Operations: Combining Information
\newcommand{\opA}{\odot}
\newcommand{\opB}{\otimes}
\newcommand{\opC}{\oplus}

% Relview Type Text
\newcommand{\relview}{\texttt{RELVIEW}}

%________________________________________________________________________
%_________________      END COVERT CHANNEL MACROS      __________________
%________________________________________________________________________

%________________________________________________________________________
%_________________       BEGIN HIGLIGHTING MACROS      __________________
%________________________________________________________________________

% Highlight Text
\newcommand{\hl}[1]{\colorbox{grey}{#1}}

% Coloured Boxes For ByteField
\newcommand{\colorbitbox}[3]
	{\rlap{\bitbox{#2}{\color{#1}\rule{\width}{\height}}}\bitbox{#2}{#3}}
\newcommand{\colorwordbox}[3]
	{\rlap{\wordbox{#2}{\color{#1}\rule{\width}{\height}}}\wordbox{#2}{#3}}

%________________________________________________________________________
%_________________        END HIGLIGHTING MACROS       __________________
%________________________________________________________________________


