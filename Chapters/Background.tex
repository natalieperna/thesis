%# Inline expansion

%Inline expansion replaces a function call site with the body of the called function

In this chapter, we introduce the necessary compiler theory and logical background concepts required for the understanding of the material presented in the thesis. In Section~\ref{sec:agda}, we give an introduction to Agda and its compiler. In Section~\ref{sec:lambda_calc}, we review some of the mathematical background useful for understanding our optimisations. Finally, in Section~\ref{sec:background_conclusion}, we conclude with a summary of the core concepts and describe where they are used throughout the remainder of the thesis.

\section{Agda}
\label{sec:agda}

Agda is a dependently typed functional programming language.

%The Agda package: https://hackage.haskell.org/package/Agda

% Include an Agda tutorial as per: http://dspace.library.uu.nl:8080/handle/1874/256628
% or is it necessary to understand thesis?

The core syntax of Agda is a dependently typed lambda calculus, with a simple grammar as shown in Figure~\ref{fig:grammar}.

\begin{figure}
\begin{align*}
a ::=~& x               & \text{variable}\\
    |~& \lambda x \to a & \text{abstraction}\\
    |~& a~a             & \text{application}\\
    |~& (x : a) \to a   & \text{function space}\\
    |~& Set[n]          & \text{universe}\\
    |~& (a)             & \text{grouping}
\end{align*}
\caption{Agda core syntax grammar.\cite{agdawiki}}
\label{fig:grammar}
\end{figure}

\subsection{Module System}
% Either before or after compiler section
% Discuss Agda module system and its translation
% Arguments inheited from all enclosing modules

\subsection{Compiler}

The Agda programming language's first and most-used backend is MAlonzo, or more generically, the GHC (Glasgow Haskell Compiler) backend.\cite{benke2007} Given an Adga module containig a \lstinline{main} function, the Agda \texttt{--compile} option will compile the program using the GHC backend, which translates an Agda program into Haskell source. The generated Haskell source can then be automatically or manually (with \texttt{--ghc-dont-call-ghc}) compiled to an executable program via GHC.\cite{agdadocs} % http://agda.readthedocs.io/en/latest/tools/compilers.html

Though there are several stages of translation and compilation in this process, the transition of primiary interest for our optimization is the conversion of compiled clauses to a ``treeless'' syntax, after Agda type-checking, but before Haskell source is generated.

Agda functions are implemented by giving both a type and a definition. Functions on datatypes can be defined by pattern matching on the constructors of that datatype, describing structurally recursive functions.\cite{agdawiki} % http://wiki.portal.chalmers.se/agda/agda.php?n=Docs.DatatypeAndFunctionDefinitions
Because function definitions in Agda are written as a series of one or more pattern matching clauses on possible variable inputs, we can construct an equivalent definition via case tree.\cite{agdawiki} % http://wiki.portal.chalmers.se/agda/agda.php?n=Docs.PatternMatching
Compiled clauses are, simply put, case trees.

This should sound familiar to users of functional programming languages like Haskell. Unlike Haskell, however, Agda does not permit partial functions. Therefore, functions defined by pattern matching must not exclude any possible cases from the pattern matching clauses.\cite{agdawiki} % http://wiki.portal.chalmers.se/agda/pmwiki.php?n=ReferenceManual.Totality#Coveragechecking}
Once coverage checking is completed, pattern matching can be translated into case trees by successively splitting on each variable.\cite{agdahackage} % https://hackage.haskell.org/package/Agda-2.5.2/docs/Agda-TypeChecking-CompiledClause.html
% TODO Take for example the following case tree equivalence for a foo function, defined by pattern matching, in Agda: ??? \cite{https://hackage.haskell.org/package/Agda-2.5.2/docs/Agda-TypeChecking-CompiledClause.html}

%Compiled clauses are case trees with their bodies.
%Document compiled clauses: https://hackage.haskell.org/package/Agda-2.5.2/docs/Agda-TypeChecking-CompiledClause.html

The treeless syntax is the input to the compiler backend of Agda. After all type-checking is complete, the higher-level internal syntax of compiled clauses is translated to a ``treeless'' syntax, the name for which is derived from its use of case expressions instead of case trees. The other notable difference between compiled clauses and treeless syntax is the absence of datatypes and constructors.\cite{agdahackage} %TODO WHY? % https://hackage.haskell.org/package/Agda-2.5.2/docs/Agda-Syntax-Treeless.html

\subsubsection{Treeless Syntax}

The treeless syntax is constructed from \lstinline{TTerm}s %TODO What does the T stand for?
and is a representation of the abstract syntax tree for the compiler backend. It can be reasoned about as a lambda calculus with all local variable using de Bruijn indices. A listing of \lstinline{TTerm} constructors is shown in Figure~\ref{code:TTerm}.

In this section we examine the constructors of \lstinline{TTerm}s one-by-one\cite{agdahackage}, then discuss how substitutions can be safely performed on \lstinline{TTerm}s as a preface for discussing our optimizations.

% TODO Read about how sharing is lost with substitution in the agda mailing list: https://lists.chalmers.se/pipermail/agda/2017/009379.html

\begin{figure}
\begin{lstlisting}[style=blockhaskell]
type Args = [TTerm]

data TTerm = TVar Nat
           | TPrim TPrim
           | TDef QName
           | TApp TTerm Args
           | TLam TTerm
           | TLit Literal
           | TCon QName
           | TLet TTerm TTerm
           | TCase Nat CaseType TTerm [TAlt]
           | TUnit
           | TSort
           | TErased
           | TError TError

data TAlt = TACon QName Nat TTerm
          | TAGuard TTerm TTerm
          | TALit Literal TTerm
\end{lstlisting}
\caption{\lstinline{TTerm} and \lstinline{TAlt} datatype definitions.}
\label{code:TTerm}
\end{figure}

% TODO present de bruijn indices

A \textbf{\lstinline{TVar}} is a de Bruijn indexed variable term.

A \textbf{\lstinline{TPrim}} is a compiler-related primitive, such as addition, subtraction and equality on some primitive types.

A \textbf{\lstinline{TDef}} is a qualified name identifying a function or datatype definition % TODO confirm

A \textbf{\lstinline{TApp}} is a \lstinline{TTerm} applied to a list of arguments, where each argument is itself a \lstinline{TTerm}.

A \textbf{\lstinline{TLam}} % TTerm

A \textbf{\lstinline{TLit}} % Literal

A \textbf{\lstinline{TCon}} % QName

A \textbf{\lstinline{TLet}} introduces a new local term binding in a term body.           

A \textbf{\lstinline{TCase}} is a case expression on a case scrutinee (always a de Bruijn indexed variable), a case type, a default value and a list of alternatives.

The case alternatives, \textbf{\lstinline{TAlt}}s, may be constructed from:
\begin{itemize}
\item a \lstinline{TACon}, which matches on a constructor of a given qualified name, biding the appropriate number of pattern variables to the body term if a match is made. Note that a \lstinline{TCase}'s list of \lstinline{Args} must have unique qualified names for each \lstinline{TACon}.
\item a \lstinline{TAGuard}, which matches on a boolean guard and binds no variables if matched against.
\item a \lstinline{TALit}, which matches on a literal term.
\end{itemize}
          
A \textbf{\lstinline{TUnit}} is used for levels. % TODO what does that mean

A \textbf{\lstinline{TSort}}

A \textbf{\lstinline{TErased}}

A \textbf{\lstinline{TError}} is used to indicate a runtime error.

\section{Lambda Calculus}
\label{sec:lambda_calc}

In order to perform the desired optimisations on the abstract syntax tree, we must be able to perform substitutions on terms. Treating the \lstinline{TTerm} structure as a specialized $\lambda$-calculus, we can implement substitution as a function on terms.

Because our terms are built on de Bruijn indexed variables, we use the explicit substitution of a $\lambda\sigma$-calculus as a reference for understanding correct substitution on terms in the context of local variables bound by incrementing indices. The $\lambda\sigma$-calculus is a refinement of the $\lambda$-calculus where substitutions are manipulated explicitly, and substitution application is a term constructor rather than a meta-level notation.\cite{abadi1991}

\subsection{De Bruijn indices}

In order to eliminate the need for named variables in $\lambda$-calculus notation, de Bruijn notation is used to represent bound terms (variables) with natural numbers. In any term, the positive integer $n$ refers to the $n$th surrounding $\lambda$ binder.\cite{debruijn1972}

% TODO Graphical example like this: https://en.wikipedia.org/wiki/De_Bruijn_index#/media/File:De_Bruijn_index_illustration_1.svg

Take then, for instance, a simple case of the classical application of the $\beta$-rule (See Figure~\ref{eq:beta_rule}). Beta reduction is the process of simplifying an application of a function to the resulting substituted term. However, in order to $\beta$-reduce $(\lambda a)b$, we must not only substitute $b$ into the appropriate occurrences in $a$. As the $\lambda$ binding disappears, we must also decrement all remaining free indices in $a$. This adapted form of the $\beta$-rule can be represented by the infinite substitution showin in Figure~\ref{eq:beta_rule2}).\cite{abadi1991}

% TODO Combine beta rule figures into subfigures a and b in a single figure

\begin{figure}
\begin{equation*}
(\lambda x.t)s \to_{\beta} t[x := s]
\end{equation*}
\caption{The classical $\beta$-reduction rule.}
\label{eq:beta_rule}
\end{figure}

\begin{figure}
\begin{equation*}
(\lambda t)s \to_{\beta} t[1 := s, 2 := 1, 3 := 2, ...]
\end{equation*}
\caption{The modified $\beta$-reduction rule for de Bruijn notation.}
\label{eq:beta_rule2}
\end{figure}

However, the substitution in this adapted rule must be evaluated carefully to produce a correct result. Consider if the term $t$ contains another $\lambda$ binding. As the substitution is applied to that nested $\lambda$ term, occurences of $1$ should not be replaced with $s$, because occurrences of $1$ refer to the nested $\lambda$ term's bound variable. Instead, occurrences of $2$ should be replaced with $s$; likewise, occurrences of $3$ should be replaced by $2$, and so on. We thus ``shift'' the substitution.\cite{abadi1991}

It is also important when applying substitutions to $\lambda$ terms that we avoid the unintended capture of free variables in our terms being substituted in. Imagine again the nested $\lambda$ term, with occurences of $2$ being replaced with $s$. Occurences of $1$ in $s$ must be replaced with $2$, else the nested $\lambda$ binder will capture the index. We this ``lift'' the indices of $s$. These two caveats result in the substitution rule in Figure~\ref{eq:debruijn_sub}.\cite{abadi1991}

\begin{figure}
\begin{equation*}
(\lambda t)[1 := s, 2 := 1, ...] = \lambda t[2 := s[1 := 2, 2 := 3, ...], 3 := 2, ...]
\end{equation*}
\caption{The substitution rule for de Bruijn indexed lambda terms.}
\label{eq:debruijn_sub}
\end{figure}

Recognizing the required index ``shifting'' and ``lifting'' in the Figure~\ref{eq:debruijn_sub} substitution rule should suffice as background for understanding the variable manipulation performed in our optimisation.

% Here has a good introduction that may be a good reference: http://fi.ort.edu.uy/innovaportal/file/5328/3/formalisationconstructivetypetheorystoughtonsubstitution.pdf

\section{Conclusion}
\label{sec:background_conclusion}
% Begin Section

The objective of this chapter is to give readers the required mathematical background of our approach. We have presented ABC and EFG since we will be ...  will be discussed further in Chapter~\ref{cha:main_chapter}.

% End Section
