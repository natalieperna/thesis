In this chapter, we introduce the necessary mathematical concepts required for the understanding of the material presented in the thesis. In Section~\ref{sec:section_one}, we give an introduction to ABC. In Section~\ref{sec:section_two}, we give definitions and examples of EFG. Finally, in Section~\ref{sec:background_conclusion}, we conclude with a summary of the core concepts and describe where they are used throughout the remainder of the thesis.

\section{Section One}
\label{sec:section_one}
% Begin Section

Donec scelerisque risus et sollicitudin consectetur. In ac mattis lorem. Quisque commodo odio dui, a iaculis metus tristique et. Suspendisse sit amet interdum nunc, pretium gravida metus. Suspendisse a erat egestas, egestas mi a, tincidunt mi. Vivamus justo orci, porttitor ac tincidunt sit amet, fringilla eget neque. Integer at cursus quam. Ut malesuada ullamcorper erat finibus vestibulum. Nulla ullamcorper, enim ac accumsan placerat, lorem magna consequat nulla, nec pulvinar lacus neque vel quam. Donec posuere velit in leo vestibulum bibendum. Sed scelerisque magna vel orci tempor, vel bibendum nulla consequat. Maecenas varius nec leo eget volutpat. Vivamus vulputate mollis blandit. Nulla ligula risus, tincidunt nec dolor ac, ornare ultricies risus~\cite{Gries1993aa}. \newline

Mauris fermentum, nibh mollis cursus congue, orci velit imperdiet libero, ac mollis nunc nunc ut ex. Etiam tellus nibh, pellentesque et turpis ac, aliquet maximus turpis. Etiam ornare justo ac sodales rutrum. Morbi sed tellus enim. Donec malesuada ante tortor, sit amet tempus turpis molestie venenatis. Sed vitae eros vitae elit ultricies tincidunt. Nullam quis velit semper nisi ullamcorper ullamcorper eu id elit. Cras malesuada vehicula purus, et mattis turpis volutpat eu.

% End Section

\section{Section Two}
\label{sec:section_two}
% Begin Section

Donec scelerisque risus et sollicitudin consectetur. In ac mattis lorem. Quisque commodo odio dui, a iaculis metus tristique et. Suspendisse sit amet interdum nunc, pretium gravida metus. Suspendisse a erat egestas, egestas mi a, tincidunt mi. Vivamus justo orci, porttitor ac tincidunt sit amet, fringilla eget neque. Integer at cursus quam. Ut malesuada ullamcorper erat finibus vestibulum. Nulla ullamcorper, enim ac accumsan placerat, lorem magna consequat nulla, nec pulvinar lacus neque vel quam. Donec posuere velit in leo vestibulum bibendum. Sed scelerisque magna vel orci tempor, vel bibendum nulla consequat. Maecenas varius nec leo eget volutpat. Vivamus vulputate mollis blandit. Nulla ligula risus, tincidunt nec dolor ac, ornare ultricies risus~\cite{Schmidt1993aa}. \newline

Mauris fermentum, nibh mollis cursus congue, orci velit imperdiet libero, ac mollis nunc nunc ut ex. Etiam tellus nibh, pellentesque et turpis ac, aliquet maximus turpis. Etiam ornare justo ac sodales rutrum. Morbi sed tellus enim. Donec malesuada ante tortor, sit amet tempus turpis molestie venenatis. Sed vitae eros vitae elit ultricies tincidunt. Nullam quis velit semper nisi ullamcorper ullamcorper eu id elit. Cras malesuada vehicula purus, et mattis turpis volutpat eu.
\begin{definition}
\label{def:cartesianproduct}
	Given two sets, $A$ and $B$, we define the Cartesian product\index{cartesian product} $A \times B$ as
	\begin{equation*}
		A \times B = \sets{(x,y)}{x \in A \nAnd y \in B}
	\end{equation*}
\end{definition}

% End Section

\section{Conclusion}
\label{sec:background_conclusion}
% Begin Section

The objective of this chapter is to give readers the required mathematical background of our approach. We have presented ABC and EFG since we will be ...  will be discussed further in Chapter~\ref{cha:main_chapter}.

% End Section
