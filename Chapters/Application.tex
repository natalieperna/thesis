In this chapter, we look at the effect of the optimisation techniques formulated in Chapter~\ref{cha:main_chapter} on different Agda code samples. Through a series of examples, we will see the effect of these optimisations and the degree to which they can yield increases in efficiency for both memory allocation space and execution time.

\section{Inlining a Simple Module}

\input{Figures/Agda/latex/Inline1}

Take for example the simple usage of record projections in Figure~\ref{code:inline1}. When we compile this module once without \texttt{--inline-proj} on, and once again with \texttt{--inline-proj} enabled, a unified \texttt{diff} of two of the generated Haskell files gives us what is shown in Figure~\ref{fig:inline1_diff}.

% TODO Replace with nicer looking diff, maybe from SourceTree?
\begin{figure}
\begin{verbatim}
--- Inline1-compile/MAlonzo/Code/Inline1.hs
+++ Inline1-compile-inline/MAlonzo/Code/Inline1.hs
@@ -31,4 +31,9 @@
       MAlonzo.Code.IO.du42 ()
       (coe
          MAlonzo.Code.IO.d150
-         (coe MAlonzo.Code.Data.Nat.Show.d22 (coe d18 d22)))
+         (coe MAlonzo.Code.Data.Nat.Show.d22
+            (let v0 = coe d22 in
+             case coe v0 of
+               C20 v1 v2 -> coe v2
+               _ -> MAlonzo.RTE.mazUnreachableError)))
\end{verbatim}

\caption{Unified difference of the \AgdaModule{Inline1} module compiled without and then with \texttt{--inline-proj}.}
\label{fig:inline1_diff}
\end{figure}

The compiled projection function \AgdaField{Pair.snd}, that is \lstinline{d18} in the Haskell code, is replaced with a Haskell expression that cases on the pair (\AgdaFunction{p} in Agda, \lstinline{d22} in Haskell) and returns the second field.

This inline expansion can yield benfits on its own in certain codebases (see Section~TODO) by creating opportunities for GHC to optimise. The resulting inlined code also gives us better opportunities for further optimization within the Agda backend, as we discuss in Sections TODO and TODO.

TODO Squash it too

\section{Inlining RATH-Agda Main}
\label{sec:app_one}
% Begin Section

RATH-Agda is a basic category and allegory theory library developed by Kahl, et al.\cite{kahl2017} It includes theories relating to semigroupoids, division allegories, typed Kleene algebras and monoidal categories, among other topics.\cite{kahl2017} The RATH-Agda repository also provides a set of test cases in a \AgdaModule{Main} module, which can be used to test a variety of typical uses of the library's functions.

\subsection{Before}

\begin{figure}
\begin{verbatim}
        Time and Allocation Profiling Report  (Final)

           Main +RTS -S -H7G -M7G -A128M -p -RTS

        total time  =        6.75 secs   (6755 ticks @ 1000 us, 1 processor)
        total alloc = 1,300,428,688 bytes  (excludes profiling overheads)

COST CENTRE                                                 %time %alloc

Data.Product.Σ.proj₂                                         10.4    0.0
Data.Product.Σ.proj₁                                          7.2    0.0
Categoric.KleeneCategory.DirectSum.SumStar.Square.E⋆′         4.9    7.0
Data.SUList.ListSetMap.RawLSM.RawLSM3.RawLSM3-comp.comp       4.1   10.4
Data.SUList.ListSetMap.RawLSM.RawLSM3.RawLSM3-comp.comp₀      3.9    3.7
...
\end{verbatim}
\caption{Profiling report for RATH-Agda's \AgdaModule{Main} module, compiled without projection inlining.}
\label{fig:main_prof}
\end{figure}


In profiling the runtime of this \AgdaModule{Main} module, we found that an inordinate amount of time was spent on evaluating simple record projections. The first few lines of the profiling report in Figure~\ref{fig:main_prof} indicate that the greatest cost centres in terms of time are the two simple record projections for the $\Sigma$ data type, with a combined 17.6\% of execution time spent evaluating them.

Because enabling profiling does have an affect on execution, we also re-compiled the module without profiling and ran it six times, measuring execution time with the Unix \texttt{time} command, to determing its average runtime as 1.60 seconds.

\subsection{After}

\begin{figure}[h!]
\begin{verbatim}
        Time and Allocation Profiling Report  (Final)

           Main +RTS -S -H7G -M7G -A128M -p -RTS

        total time  =        5.26 secs   (5261 ticks @ 1000 us, 1 processor)
        total alloc = 1,299,709,408 bytes  (excludes profiling overheads)

COST CENTRE                                                 %time %alloc

Data.SUList.ListSetMap.FinRel.Utils.FinId                     6.2    8.6
Data.SUList.ListSetMap.RawLSM.RawLSM3.MapImage.mapImage₀      5.6    5.5
Data.SUList.ListSetMap.RawLSM.RawLSM3.RawLSM3-comp.comp       4.9    8.8
Categoric.KleeneCategory.DirectSum.SumStar.Square.E⋆′         4.7    6.6
Data.SUList.ListSetMap.Semigroupoid.LSMJoinOp.\               3.8    8.3
...
\end{verbatim}
\caption{Profiling report for RATH-Agda's \AgdaModule{Main}~module, compiled with projection inlining enabled.}
\label{fig:main_inline_prof}
\end{figure}


By compiling \AgdaModule{Main} with our new option, \texttt{--inline-proj}, enabled, we reduced total runtime and memory allocation, as can be seen by comparing Figure~\ref{fig:main_inline_prof} and Figure~\ref{fig:main_prof}.

We again re-compiled the module without profiling and ran it six times, measuring execution time with the Unix \texttt{time} command, to determing its average runtime with projections inlined as 1.44 seconds. We therefore produced a speedup of 1.11$\times$.

% End Section

\section{Conclusion}
\label{sec:application_conclusion}
% Begin Section

In imperdiet purus nec eleifend finibus. Aliquam non tempor massa. Etiam ac felis et ante varius vehicula nec eget tortor. Proin posuere quis felis non rutrum. Aenean quis felis ut ex sagittis pellentesque sit amet tempus nisl. Nam nec tellus ut lorem posuere semper non ac arcu. Nulla faucibus purus libero, in pellentesque sapien commodo tristique. Etiam consectetur lectus elit, id porttitor justo dignissim interdum. Donec ut nisl metus. Nulla sed dui lacus. Donec tristique dignissim massa sed ultricies. Maecenas iaculis arcu diam, ut dictum nisi euismod vitae. Praesent id imperdiet augue.

% End Section
