\chapter{Conclusion}
\label{cha:conclusion}

In this chapter, we discuss various aspects of our optimisations in summary. In Section~\ref{sec:assessment_of_the_contributions}, we assess the strengths and weaknesses of the main contributions. In Section~\ref{sec:future_work}, we discuss future work that could follow this project. Finally, in Section~\ref{sec:closing_remarks}, we draw conclusions from the thesis and give some closing remarks.

\section{Assessment of the Contributions}
\label{sec:assessment_of_the_contributions}

\subsection{Strengths of the Contributions}
\label{sub:strengths_of_the_contributions}

The contributions described herein have a number of strengths that will serve the Agda development community.

Primarily, as shown by the results of the applications of our optimisations in Section~\ref{cha:application_of_main}, our optimisations will reduce the runtime execution and heap allocation requirements of many Agda programs with a negligible impact on compile time, and do not show adverse effects in any of the tests we have performed.

Secondarily, this thesis may also serve as a detailed documentation of the portions of the Agda compiler and GHC backend that are necessary to understand for incorporating future optimising transformations. We hope that future contributors to Agda will find this presentation of our study of the Agda compiler useful in implementing their own desired optimisations.

\subsection{Weaknesses of the Contributions}
\label{sub:weaknesses_of_the_contributions}

Althought the net effect of our optimisations is a positive one, there are still a number of weaknesses that warrant consideration.

A clear weakness of our case squashing optimisation is its isolated implementation. Because it was developed as an independent transformation, it requires an additional traversal of the treeless terms to execute. Further, some of the logic built to deal with the handling of de Bruijn indices would have been avoidable had it been built as part of an existing set of optimisations that had similar optimisation helper functions already developed. As presented in Subsection~\ref{sub:alternate_case_squash}, an independent version of case squashing has since been developed and introduced into the Agda compiler, as part of the \lstinline{Agda.Compiler.Treeless.Simplify} module, which addreseses both of these weaknesses.

Both the projection inlining and case squashing optimisations make use of accumulated environment parameters, which can be handled more modularly and appropriately using monads. This potential for code refactoring is discussed further in Section~\ref{sec:future_work}.

\section{Future Work}
\label{sec:future_work}

In our implementations of projection inlining and case squashing, it was noted that environments of relevant information were carried through graph traversal. For projection inlining, this was an environment of previously inlined projections is maintained to avoid looping on recursive inlining calls. For case squashing, this was an environment of previously met cases expressions. These environments are currently maintained as a list objects, passed from function call to function call. For modularity and maintainability of the code, these environments would be better refactored into reader monad transformers, which would allow an inherited environment to be bound to the function results and passed through to subcomputations via the given monad.

Further testing of the projection inlining optimisation on a greater variety of Agda codebases is also a necessary next step before it can be safely integrated with a stable release branch of the compiler.

\section{Closing Remarks}
\label{sec:closing_remarks}

We have implemented, tested, and profiled a series of optimisations to the Agda compiler which improve execution time and reduce memory usage for many of the Agda programs tested, and have no negative performance effects in any of our tests. Our optimisations have a negligible impact on compile time, and serve previously unmet needs of our team as well as many other Agda developers.
