In this chapter, we introduce dependently typed functional programming. In Section~\ref{sec:general_context}, we give an introduction to the Agda programming language and explain its history, advantages, and potential future. In Section~\ref{sec:specific_context}, we introduce the Agda compiler and GHC. In Section~\ref{sec:literature_survey}, we provide a review of the literature and discuss some existing techniques for improving functional code with optimising compilers and indicating how the existing techniques are not sufficient to achieve the desired performance in our use cases. In Section~\ref{sec:motivation}, we give the motivation for the new optimisation strategies introduced here. In Section~\ref{sec:problem_statement}, we state the problem subject of our work. In Section~\ref{sec:main_contributions}, we summarize our contributions, namely the optimisation strategies and their implementation in the Agda backend. Finally, in Section~\ref{sec:structure_of_the_thesis}, we give the structure of the remainder of the thesis.

\section{General Context}
\label{sec:general_context}
% Begin Section
% In Section~\ref{sec:general_context}, we give an introduction to the Agda programming language and explain its history, advantages, and potential future.

Lorem ipsum dolor sit amet, consectetur adipiscing elit. Sed posuere eros ac nisi feugiat commodo. Curabitur sit amet ex feugiat, efficitur felis ut, mollis massa. Aliquam odio quam, tincidunt ut efficitur nec, condimentum in odio. Duis sagittis in risus id laoreet. Etiam ut nisi odio. Sed eu convallis felis. Nullam eget libero metus. Ut sed leo a nibh tempor vestibulum. Mauris leo nibh, consequat sed vehicula ac, rhoncus sit amet ligula. Vestibulum efficitur ornare ullamcorper. In rhoncus mollis condimentum. \newline

Lorem ipsum has three facets which have a strong relationship to ABC: confidentiality, integrity, and availability~\cite{Bishop2002aa}. Confidentiality\index{confidentiality} refers to consectetur adipiscing elit. Sed posuere eros ac nisi feugiat commodo. Curabitur sit amet ex feugiat, efficitur felis ut, mollis massa. Aliquam odio quam, tincidunt ut efficitur nec, condimentum in odio. Duis sagittis in risus id laoreet. Etiam ut nisi odio. Integrity\index{integrity} refers to consectetur adipiscing elit. Sed posuere eros ac nisi feugiat commodo. Curabitur sit amet ex feugiat, efficitur felis ut, mollis massa. Aliquam odio quam, tincidunt ut efficitur nec, condimentum in odio. Duis sagittis in risus id laoreet. Etiam ut nisi odio. Availability\index{availability} refers to the consectetur adipiscing elit. Sed posuere eros ac nisi feugiat commodo. Curabitur sit amet ex feugiat, efficitur felis ut, mollis massa. Aliquam odio quam, tincidunt ut efficitur nec, condimentum in odio. Duis sagittis in risus id laoreet. Etiam ut nisi odio. Confidentiality, integrity, and availability are strongly related to consectetur adipiscing elit. Sed posuere eros ac nisi feugiat commodo. Curabitur sit amet ex feugiat, efficitur felis ut, mollis massa. Aliquam odio quam, tincidunt ut efficitur nec, condimentum in odio. Duis sagittis in risus id laoreet. Etiam ut nisi odio. \newline

% End Section

\section{Specific Context}
\label{sec:specific_context}
% Begin Section
% In Section~\ref{sec:specific_context}, we introduce the Agda compiler and GHC.

Lorem ipsum dolor sit amet, consectetur adipiscing elit. Sed posuere eros ac nisi feugiat commodo. Curabitur sit amet ex feugiat, efficitur felis ut, mollis massa. Aliquam odio quam, tincidunt ut efficitur nec, condimentum in odio. Duis sagittis in risus id laoreet. Etiam ut nisi odio. Sed eu convallis felis. Nullam eget libero metus. Ut sed leo a nibh tempor vestibulum. Mauris leo nibh, consequat sed vehicula ac, rhoncus sit amet ligula. Vestibulum efficitur ornare ullamcorper. In rhoncus mollis condimentum. \newline

Lorem ipsum dolor sit amet, consectetur adipiscing elit. Sed posuere eros ac nisi feugiat commodo. Curabitur sit amet ex feugiat, efficitur felis ut, mollis massa. Aliquam odio quam, tincidunt ut efficitur nec, condimentum in odio. Duis sagittis in risus id laoreet~\cite{Lampson1973aa}. Etiam ut nisi odio. Sed eu convallis felis. Nullam eget libero metus. Ut sed leo a nibh tempor vestibulum~\cite{Kemmerer1983aa}. Mauris leo nibh, consequat sed vehicula ac, rhoncus sit amet ligula. Vestibulum efficitur ornare ullamcorper. In rhoncus mollis condimentum. \newline

% End Section

\section{Literature Survey of XYZs}
\label{sec:literature_survey}
% Begin Section
% In Section~\ref{sec:literature_survey}, we provide a review of the literature and discuss some existing techniques for improving functional code with optimising compilers and indicating how the existing techniques are not sufficient to achieve the desired performance in our use cases.

Lorem ipsum dolor sit amet, consectetur adipiscing elit. Sed posuere eros ac nisi feugiat commodo. Curabitur sit amet ex feugiat, efficitur felis ut, mollis massa. Aliquam odio quam, tincidunt ut efficitur nec, condimentum in odio. Duis sagittis in risus id laoreet. Etiam ut nisi odio~\cite{Goguen1982aa}. Sed eu convallis felis. Nullam eget libero metus. Ut sed leo a nibh tempor vestibulum. Mauris leo nibh, consequat sed vehicula ac, rhoncus sit amet ligula. Vestibulum efficitur ornare ullamcorper. In rhoncus mollis condimentum. \newline

Lorem ipsum dolor sit amet, consectetur adipiscing elit. Sed posuere eros ac nisi feugiat commodo. Curabitur sit amet ex feugiat, efficitur felis ut, mollis massa. Aliquam odio quam, tincidunt ut efficitur nec, condimentum in odio. Duis sagittis in risus id laoreet. Etiam ut nisi odio. Sed eu convallis felis. Nullam eget libero metus. Ut sed leo a nibh tempor vestibulum. Mauris leo nibh, consequat sed vehicula ac, rhoncus sit amet ligula. Vestibulum efficitur ornare ullamcorper. In~\cite{Nagatou2006aa} rhoncus mollis condimentum. \newline

% End Section

\section{Motivation}
\label{sec:motivation}
% Begin Section
% In Section~\ref{sec:motivation}, we give the motivation for the new optimisation strategies introduced here.

Sed sit amet elementum ligula. Vivamus faucibus, augue vel tincidunt elementum, sapien est ullamcorper nisl, non gravida urna massa a eros. Nulla sit amet feugiat leos~\cite{Gray2000aa}. Pellentesque id ante consectetur purus imperdiet tincidunt a tincidunt diam. Mauris sagittis sollicitudin cursus. Morbi elementum arcu quis mollis gravida. Mauris id accumsan libero. Morbi scelerisque id nibh eget venenatis. Curabitur iaculis mi at urna varius malesuada. In luctus risus non justo hendrerit commodo. \newline

Proin ultricies enim sit amet libero fermentum convallis. Nunc blandit mauris ante, non vulputate justo feugiat at. Sed a cursus eros. In condimentum massa leo, vel sollicitudin eros egestas nec. Donec in tincidunt ex, sit amet fermentum est. Maecenas maximus lacus auctor tellus fermentum, in dapibus quam efficitur. Nunc pellentesque suscipit purus, ac tempus urna. Praesent nec libero luctus, faucibus velit eget, malesuada turpis. In commodo porta odio at aliquam. Suspendisse molestie dui at lacus suscipit, in auctor augue ultrices. In euismod fermentum justo, eget rutrum ligula gravida porta. Nam eu rhoncus nulla. Sed faucibus enim libero, non venenatis leo pharetra nec. Mauris et eros quis quam condimentum vulputate. Pellentesque nisl diam, ultrices non facilisis lobortis, hendrerit vitae turpis. \newline


% End Section

\section{Problem Statement}
\label{sec:problem_statement}
% Begin Section
% In Section~\ref{sec:problem_statement}, we state the problem subject of our work.

Etiam erat odio, tempor vel mi eu, porta pretium lacus. Fusce dictum faucibus porttitor. Nunc vulputate mauris sed odio aliquet volutpat. Integer lorem dolor, volutpat a sagittis vitae, efficitur at turpis. Phasellus sodales tortor ac nunc tincidunt lobortis. Mauris dictum auctor nibh, ut aliquam dolor feugiat eu. Nullam pellentesque urna sed mauris fringilla, vel ornare nunc ornare. Nullam ac sollicitudin arcu. Nullam a accumsan orci. \newline

Etiam erat odio, tempor vel mi eu, porta pretium lacus. Fusce dictum faucibus porttitor. Nunc vulputate mauris sed odio aliquet volutpat. Integer lorem dolor, volutpat a sagittis vitae, efficitur at turpis. Phasellus sodales tortor ac nunc tincidunt lobortis. Mauris dictum auctor nibh, ut aliquam dolor feugiat eu. Nullam pellentesque urna sed mauris fringilla, vel ornare nunc ornare. Nullam ac sollicitudin arcu. Nullam a accumsan orci. \newline

% End Section

\section{Main Contributions}
\label{sec:main_contributions}
% Begin Section
% In Section~\ref{sec:main_contributions}, we summarize our contributions, namely the optimisation strategies and their implementation in the Agda backend.

The main contributions to the XYZ include:
\begin{enumerate}[(i)]
	\item Cum sociis natoque penatibus et magnis dis parturient montes, nascetur ridiculus mus.
	\item Curabitur rutrum non lacus id tristique. Duis efficitur condimentum risus ac egestas.
	\item  Mauris id dapibus quam, non mollis purus. Ut vel iaculis sem. Morbi erat urna, cursus nec vehicula et, finibus ut nisi. Aenean eget nulla dui.
\end{enumerate}

% End Section

\section{Structure of the Thesis}
\label{sec:structure_of_the_thesis}
% Begin Section
% Finally, in Section~\ref{sec:structure_of_the_thesis}, we give the structure of the remainder of the thesis.

The remainder of this thesis is organized as follows:

\paragraph{Chapter~\ref{cha:mathematical_background}} introduces the required mathematical background including MNO theory.

\paragraph{Chapter~\ref{cha:lit_survey}} provides a number of examples of ways in which XYZ.

\paragraph{Chapter~\ref{cha:main_chapter}} describes the process by which we formulate a new technique to ABC.

\paragraph{Chapter~\ref{cha:application_of_main}} gives a number of illustrative examples demonstrating the application of the proposed XYZ.

\paragraph{Chapter~\ref{cha:discussion}} discusses the impact of our approach in helping to remedy the problem of XYZ.

\paragraph{Chapter~\ref{cha:conclusion_and_future_work}} draws conclusions and suggests future work.

% End Section
