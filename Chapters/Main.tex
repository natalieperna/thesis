\chapter{Main Chapter}
\label{cha:main_chapter}

In this chapter, we discuss the design and implementation of our optimisations to the Agda compiler. In Section~\ref{sec:logical_representation}, we give a logical representation of the optimisations. In Section~\ref{sec:implementation}, we present our implementation, with references to source code in the Appendix. In Section~\ref{sec:usage}, we present the usage instructions for these features in our compiler branch.

\section{Logical Representation}
\label{sec:logical_representation}

We present in Figure~\ref{fig:treeless_grammar} a simplified logical representation of the Agda treeless syntax as a grammar. We use this simplification to discuss the optimisations at a logical level of abstraction. Note that variables are represented only by their de Bruijn index.

\begin{figure}[h]
\begin{align*}
t ::=~& i & \text{variable}\\
|~& d & \text{function or datatype name}\\
|~& t~t^* & \text{application}\\
|~& \lambda~0 \to t & \text{lambda abstraction}\\
|~& l & \text{literal}\\
|~& \mathtt{let}~0 = t~\mathtt{in}~t & \text{let}\\
|~& \mathtt{case}~i : \tau~\mathtt{of}~a^*~\mathtt{otherwise} \to t& \text{case}\\
\\
a ::=~& d~(ar-1)..0 \to t\\
|~& l \to t\\
\\
i \in~& \mathbb{N} & \text{de Bruijn index}\\
d \in~& \{\text{function or datatype name}\}\\
l \in~& \{\text{literal}\}\\
\end{align*}
\caption{Simplified representation of the Agda treeless syntax grammar.}
\label{fig:treeless_grammar}
\end{figure}

\subsection{Inlining Projections}

The logical representation of inlining is fairly straightforward. We recurse through the treeless representation of an Agda module. For every application of a function or datatype to a list of arguments, that is $d~t^*$, where $d$ is the name of a function or datatype, and $t$s are treeless terms, we replace $d~t^*$ with the function or datatype definition corresponding to $d$ and substitute in the $t^*$ arguments.

\subsection{Case Squashing}

The goal of case squashing is to eliminate case expressions where the scrutinee has already been matched on by an enclosing ancestor case expression. For example, given the following simplified treeless expression:

\begin{lstlisting}[style=math]
$\lambda~\textcolor{red}{0} \to$
$\lambda~0 \to$
$\lambda~0 \to$
case $\textcolor{red}{2}$ of
  $da~2~1~0 \to$
    case $0$ of
      $db~1~0 \to$
        case $\textcolor{red}{7}$ of
          $da~2~1~0 \to r$
          ...
      ...
  ...
\end{lstlisting}

we can follow the de Bruijn indices to their matching variable bindings, to see that the first and third case expressions are scrutinizing the same variable, the one bound by the outermost $\lambda$ function. Therefore, with only static analysis of the expression tree, we know that the third case expression must follow the $da~2~1~0$ branch, and we can thus safely transform our expression into the following:

\begin{lstlisting}[style=math]
$\lambda~0 \to$
$\lambda~0 \to$
$\lambda~0 \to$
case $2$ of {
  $da~\textcolor{red}{2}~\textcolor{blue}{1}~\textcolor{green}{0} \to$
    case $0$ of
      $db~1~0 \to$
        $r[2 := \textcolor{red}{4}, 1 := \textcolor{blue}{3}, 0 := \textcolor{green}{2}]$
      ...
  ...
\end{lstlisting}

We perform this ``case squashing'' by accumulating an environment of all previously scrutinized variables as we traverse the tree structure (appropriately shifting de Bruijn indices in the environment as new variables are bound), and replacing case expressions that match on the same variable as an ancestor case expressions, with the corresponding case branch's body. Any variables in the body that refer to bindings in the removed branch should be replaced with references to the bindings in the matching ancestor case expression branch.

\subsection{Generating Pattern Lets}

We can avoid generating certain trivial case expressions by identifying expressions like the following:

\begin{lstlisting}[style=math]
let $0$ = $e$
in case $0$ of
  $da~2~1~0 \to t$
	otherwise $\to X$
\end{lstlisting}

In the above example, the body of a \lstinline{let} expression is a \lstinline{case} expression scrutinizing the variable just bound by the enclosing \lstinline{let} where only one case alternative exists, and the default case is marked as \textit{unreachable}. TODO explain Unreachable markers

Our treeless syntax does not support pattern matching, but when these cases are identified before transforming into Haskell expressions, we can replace them with ``pattern lets'' like the following:

\begin{lstlisting}
let v0@(C0 v1 v2 v3) = e
in t
\end{lstlisting}

removing an unnecessary case expression, and immediately binding the appropriate constructor parameters in the \lstinline{let} expression.

\subsection{Let Pattern Floating}

\edcomm{NP}{Kahl: floating let patterns as described in Issue 1895, possibly including FloatingCase (which will have to end up producing irrefutable patterns on the function left-hand sides, not done yet), but definitely a separate section for cross-call-floating. See: Log/2017-04-26\_Float-across-function-calls\_notes.txt}

TODO

\section{Implementation}
\label{sec:implementation}

We discuss in this section some implementation details of our optimisation with reference to the Haskell data type of Agda's treeless representation, as shown in Figure~\ref{code:TTerm}.

\subsection{Inlining Projections}

Note that proper projections do not include projection-like functions, or record field values, i.e. projections applied to an argument.

TODO

\subsection{Case Squashing}

TODO

\subsection{Generating Pattern Lets}

TODO

Our next optimisation centres around generating these ``pattern lets''. The function \lstinline{term :: T.TTerm -> CC HS.Exp} in the Agda compiler backend transforms Agda terms into Haskell expressions. In this function, we introduced a new alternative that matches on terms with potential to be transformed into pattern lets, namely let expressions where the let body is a case expression, with a single constructor alternative, scrutinizing the variable bound by the parent let. Because these Agda terms used de Bruijn indexed variables, that means the case expression should be scrutinizing the 0 (most recently bound) variable.

\subsection{Let Pattern Floating}

TODO

\section{Usage}
\label{sec:usage}

\subsection{Inlining Projections}

We added the option:

\begin{verbatim}
--inline-proj                               inline proper projections
\end{verbatim}

to our Agda branch which, when enabled, will replace every call to a function that is a proper projection with its function body. 

\subsection{Case Squashing}

We added the option:

\begin{verbatim}
--squash-cases                              remove unnecessary case expressions
\end{verbatim}

to our Agda branch which, when enabled, will perform the case squashing optimisation described above.

\subsection{Generating Pattern Lets}

TODO

\subsection{Let Pattern Floating}

TODO
