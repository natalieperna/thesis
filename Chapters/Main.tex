\chapter{Main Chapter}
\label{cha:main_chapter}

In this chapter, we discuss the design and implementation of our optimisations to the Agda compiler. In Section~\ref{sec:logical_representation}, we give a logical representation of the optimisations. In Section~\ref{sec:implementation}, we present our implementation, with references to source code in the Appendix. In Section~\ref{sec:usage}, we present the usage instructions for these features in our compiler branch.

\section{Logical Representation}
\label{sec:logical_representation}

We present in Figure~\ref{fig:treeless_grammar} a simplified logical representation of the Agda treeless syntax as a grammar. We use this simplification to discuss the optimisations at a logical level of abstraction. Note that variables are represented only by their de Bruijn index.

\begin{figure}[h!]
\begin{align*}
t ::=~& i & \text{variable}\\
|~& d & \text{function or datatype name}\\
|~& t~t^* & \text{application}\\
|~& \lambda~0 \to t & \text{lambda abstraction}\\
|~& l & \text{literal}\\
|~& \mathtt{let}~0 = t~\mathtt{in}~t & \text{let}\\
|~& \mathtt{case}~i : \tau~\mathtt{of}~a^*~\mathtt{otherwise} \to t& \text{case}\\
\\
a ::=~& d~(ar-1)~..~0 \to t & \text{constructor alternative}\\
|~& l \to t & \text{literal alternative}
\end{align*}
\caption{Simplified representation of the Agda treeless syntax grammar.}
\label{fig:treeless_grammar}
\end{figure}

%i \in~& \mathbb{N} & \text{de Bruijn index}\\
%d \in~& \{\text{function or datatype name}\}\\
%l \in~& \{\text{literal}\}\\

\subsection{Inlining Projections}

The logical representation of inlining is fairly straightforward. We recurse through the treeless representation of an Agda module. For every application of a function or datatype to a list of arguments, that is $d~t^*$, where $d$ is the name of a function or datatype, and $t$s are treeless terms, we replace $d~t^*$ with the function or datatype definition corresponding to $d$ and substitute in the $t^*$ arguments.

\subsection{Case Squashing}

The goal of case squashing is to eliminate case expressions where the scrutinee has already been matched on by an enclosing ancestor case expression. Figure~\ref{fig:case_squash_rule} shows the tranformation from a case expression with repeated scrutinisations on the same variable, to the optimised ``case squashed'' version.

\begin{figure}[h]
\centering
\begin{subfigure}{.47\textwidth}
  \centering
  \begin{lstlisting}[style=math]
  case $i : \tau$ of
   $d~(ar-1)~..~0 \to$
     ...
       case $j : \tau$ of
         $d~(ar-1)~..~0 \to r$
  \end{lstlisting}
  where \lstinline[style=math]{lookupVar($i$) = lookupVar($j$)}.
\end{subfigure}
{\large$\to$}
\begin{subfigure}{.47\textwidth}
  \centering
  \begin{lstlisting}[style=math]
  case $i : \tau$ of
   $d~(ar-1)~..~0 \to$
     ...
       $r'$
  \end{lstlisting}
  where $r'$ is $r$ with variables $0, ..., (ar-1)$ replaced with the corresponding bound variables from the ancestor case expression.
\end{subfigure}
\caption{Case squashing rule.}
\label{fig:case_squash_rule}
\end{figure}

\edcomm{NP}{Define lookupVar}

For example, given the following simplified treeless expression:

\begin{lstlisting}[style=math]
$\lambda~\textcolor{red}{0} \to$
$\lambda~0 \to$
$\lambda~0 \to$
case $\textcolor{red}{2}$ of
  $da~2~1~0 \to$
    case $0$ of
      $db~1~0 \to$
        case $\textcolor{red}{7}$ of
          $da~2~1~0 \to r$
          ...
      ...
  ...
\end{lstlisting}

we can follow the de Bruijn indices to their matching variable bindings, to see that the first and third case expressions are scrutinizing the same variable, the one bound by the outermost $\lambda$ abstraction. Therefore, with only static analysis of the expression tree, we know that the third case expression must follow the $da~2~1~0$ branch, and we can thus safely transform our expression into the following substituted expression:

\begin{lstlisting}[style=math]
$\lambda~0 \to$
$\lambda~0 \to$
$\lambda~0 \to$
case $2$ of {
  $da~\textcolor{red}{2}~\textcolor{blue}{1}~\textcolor{green}{0} \to$
    case $0$ of
      $db~1~0 \to$
        $r[2 := \textcolor{red}{4}, 1 := \textcolor{blue}{3}, 0 := \textcolor{green}{2}]$
      ...
  ...
\end{lstlisting}

We perform this ``case squashing'' by accumulating an environment of all previously scrutinized variables as we traverse the tree structure (appropriately shifting de Bruijn indices in the environment as new variables are bound), and replacing case expressions that match on the same variable as an ancestor case expressions, with the corresponding case branch's body. Any variables in the body that refer to bindings in the removed branch should be replaced with references to the bindings in the matching ancestor case expression branch.

\subsection{Generating Pattern Lets}
\label{sub:logical_plet}

We can avoid generating certain trivial case expressions by identifying let expressions with the following attributes:
\begin{itemize}
  \item the body of the \lstinline{let} expression is a \lstinline{case} expression;
  \item the case expression is scrutinising the variable just bound by the enclosing \lstinline{let};
  \item only one case alternative exists, a constructor alternative; and
  \item the default case is marked as \textit{unreachable}.
\end{itemize}

Figure~\ref{fig:plet_rule} shows the rule for generating an optimised Haskell expression given a treeless expression with the above properties.

\begin{figure}[h]
\centering
\begin{subfigure}{.47\textwidth}
  \begin{lstlisting}[style=math]
  let $0$ = $e$
  in case $0$ of
    $d~(ar-1)~..~0 \to t$
    otherwise $\to u$
  \end{lstlisting}
  where \lstinline[style=math]{unreachable($s$) = true}.
\end{subfigure}
{\large$\to$}
\begin{subfigure}{.47\textwidth}
  \begin{lstlisting}[style=blockhaskell]
  let v0@(D v1 ... v(ar)) = e
  in t
  \end{lstlisting}
\end{subfigure}
\caption{Generating pattern lets rule.}
\label{fig:plet_rule}
\end{figure}

\edcomm{NP}{Define unreachable}

Our treeless syntax does not support pattern matching, but when these cases are identified before transforming into Haskell expressions, we can replace them with ``pattern lets'', removing an unnecessary case expression, and immediately binding the appropriate constructor parameters in the enclosing \lstinline{let} expression.

%\subsection{Let Pattern Floating}

%\edcomm{NP}{Kahl: floating let patterns as described in Issue 1895, possibly including FloatingCase (which will have to end up producing irrefutable patterns on the function left-hand sides, not done yet), but definitely a separate section for cross-call-floating. See: Log/2017-04-26\_Float-across-function-calls\_notes.txt}

%TODO

\section{Implementation}
\label{sec:implementation}

We discuss in this section some implementation details of our optimisation with reference to the Haskell data type of Agda's treeless representation, as shown in Figure~\ref{code:TTerm}.

\subsection{Inlining Projections}

It is worth noting that the only projections which we identify and inline are ``proper projections'', that is, we do not include projection-like functions, or record field values, i.e. projections applied to an argument.

The only major complication in implementing the projection inlining optimisation is accounting for the potential for recursive inlining to loop, resulting in non-termination of compilation. Therefore, when inlining projections, we maintain an environment of previously inlined projections and avoid inlining the same projection more than one level deep.

For a complete listing of our implementation of the projection inlining optimisation, refer to Appendix~\ref{app:to_treeless}. The \lstinline{Agda.Compiler.ToTreeless} module is responsible for converting Agda's internal syntax to the treeless syntax. It is during this translation that other manual forms of definition inlining are performed, so we introduce our optimisation as an additional guard on translating internal \lstinline{Def}s, which checks whether the definition is a projection, and performs inlining if it  is.

\subsection{Case Squashing}

The case squashing implementation is practically very similar to it's logical representation described above. While recursing through the treeless structure we accumulate an environment containing the relevant attributes of the case expressions in scope. As new variables are bound recursing down the structure, the indices stored in this environment are incremented accordingly.

For more details about how variable indices are replaced in the resulting term, refer to Subsection~\ref{sub:lambda_calc_subst}.

For a complete listing of our implementation of the case squashing optimisation, refer to Appendix~\ref{app:case_squash}.

\subsection{Generating Pattern Lets}

The \lstinline{Agda.Compiler.MAlonzo.Compiler} module is responsible for transforming Agda treeless terms into Haskell expressions. In the primary function for this compilation, we introduced a new alternative that matches on terms with potential to be transformed into pattern lets. In order to be a suitable candidate for this optimisation, a \lstinline{let} expression must exhibit the properties described in Subsection~\ref{sub:logical_plet}. Because these Agda terms used de Bruijn indexed variables, that means the case expression should be scrutinizing the 0 (most recently bound) variable, and the requirements can thus be represented with a pattern matching expression \lstinline{TLet _ (TCase 0 _ _ [TACon _ _ _])}, followed by a check that the default branch is unreachable.

For a complete listing of our implementation of the pattern let generating optimisation, refer to Appendix~\ref{app:compiler}.

%\subsection{Let Pattern Floating}

%TODO

\section{Usage}
\label{sec:usage}

\subsection{Inlining Projections}

We added the option:

\begin{verbatim}
--inline-proj                               inline proper projections
\end{verbatim}

to our Agda branch which, when enabled, will replace every call to a function that is a proper projection with its function body. 

\subsection{Case Squashing}

We added the option:

\begin{verbatim}
--squash-cases                              remove unnecessary case expressions
\end{verbatim}

to our Agda branch which, when enabled, will perform the case squashing optimisation described above.

\subsection{Generating Pattern Lets}

TODO

%\subsection{Let Pattern Floating}

%TODO
